% -------------------------------------------------------------------------------
\documentclass[10pt,hyperref={pdfpagelabels=false}]{beamer}
\let\Tiny=\tiny
% -------------------------------------------------------------------------------
%\usepackage{lmodern}
\usepackage[T1]{fontenc}
\usepackage{schemabloc}
\usepackage{ifthen}
\usepackage{amssymb,amsmath}
\usepackage[utf8]{inputenc}
\usepackage[english,brazil,brazilian]{babel}

%\usepackage[round, sort, authoryear]{natbib}
%\bibliographystyle{plainnat}

% ------------------------------------------------------------------------
\usepackage{listings}
%\usepackage{xcolor}

\lstdefinestyle{mystyle}{ 
    commentstyle=\color{codegreen},
    keywordstyle=\color{magenta},
    stringstyle=\color{codepurple},
    basicstyle=\ttfamily\footnotesize,
    breakatwhitespace=false,         
    breaklines=true,                 
    captionpos=b,                    
    keepspaces=true,                 
%    numbers=left,                    
%    numbersep=5pt,                  
    showspaces=false,                
    showstringspaces=false,
    showtabs=false,                  
    tabsize=2
}

\lstset{style=mystyle}

% -------------------------------------------------------------------------------
\mode<presentation> {
  \usetheme{JuanLesPins}%AnnArbor
  \useinnertheme{rounded}
  \useoutertheme[]{smoothtree}
%  \usecolortheme{wolverine}
  \usefonttheme[onlymath]{serif}
  \setbeamercovered{transparent}
  \usebeamertemplate*{logo}
}
% It makes your presentation go automatically to full screen
\hypersetup{pdfpagemode=FullScreen}

\title{Estruturas ARMA para modelar séries temporais}
\author{Rodrigo A. Romano}
\date{Maio, 2021}

\begin{document}

\frame{\titlepage}

\section[Outline]{}
\frame{\tableofcontents}


\section{Preâmbulo}


\frame{
\frametitle{Séries temporais}

Razões pelas quais podemos estar interessados em modelos de séries temporais incluem
\begin{itemize}
\item Obter informações e compreender melhor os fenômenos que regem a série temporal em questão;
\item Realizar previsões;
\item Detectar mudanças de padrão de comportamento.
\end{itemize}

Uma série temporal amostrada uniformemente pode ser representada por uma sequência de valores no domínio do tempo discreto, ou seja
\[
x_k \in \{0, 1, 2, \ldots, N\} .
\]

}


\section{A família de modelos ARMA}

\frame{
\frametitle{Modelos MA (moving average)}

Equação de diferenças:
\[
x_k = \mu + \epsilon_k + c_1 \epsilon_{k-1} + \ldots + c_{n_c} \epsilon_{k-n_c},
\]
onde $\mu$ é a média da série temporal, $c_i$ ($i=\{1,\ldots,n_c\}$) são os coeficientes do modelo e $\epsilon_k$ é idealmente uma realização de um ruído branco de distribuição normal, média nula e variância  $\sigma^2$.

\begin{block}{Exemplo: Processo MA  de primeira ordem ($c_1=0.9$) com média nula}
Equação de diferenças: $x_k = \epsilon_k + 0.9 \epsilon_{k-1}$

Diagrama de blocos:
%\vspace*{10pt}
\begin{center}
\begin{tikzpicture}
\sbEntree{E1}
\sbBloc[3]{Bloc}{$1 + 0.9q^{-1}$}{E1}
\sbRelier[$\epsilon_k$]{E1}{Bloc}
\sbSortie[3]{S1}{Bloc}
\sbRelier[$x_k$]{Bloc}{S1}
\end{tikzpicture}
\end{center}

\end{block}

}


\frame{
\frametitle{Modelos AR (auto--regressive)}

Equação de diferenças:
\[
x_k = a_1 x_{k-1} + \ldots + a_{n_a} x_{k-n_a} + \epsilon_k,
\]
onde $a_i$ ($i=\{1,\ldots,n_a\}$) são os coeficientes do modelo e $\epsilon_k$ é idealmente uma realização de um ruído branco de distribuição normal, média nula e variância  $\sigma^2$.

\begin{block}{Exemplo: Processo AR de segunda ordem com média nula}

Equação de diferenças: $x_k = 1.5 x_{k-1} - 0.7 x_{k-2} + \epsilon_k$

Diagrama de blocos:
%\vspace*{10pt}
\begin{center}
\begin{tikzpicture}
\sbEntree{E1}
\sbBloc[3]{Bloc}{$\dfrac{1}{1 -1.5q^{-1}+ 0.7q^{-2}}$}{E1}
\sbRelier[$\epsilon_k$]{E1}{Bloc}
\sbSortie[3]{S1}{Bloc}
\sbRelier[$x_k$]{Bloc}{S1}
\end{tikzpicture}
\end{center}
Pólos complexos: $q_{1,2}= 0.75 \pm 0.371i$
\end{block}

}


\frame{
\frametitle{Modelos ARMA}

Podemos combinar os termos MA e AR para obter o modelo ARMA:
\begin{align*}
x_k = & a_1 x_{k-1} + \ldots + a_{n_a} x_{k-n_a} \\
	& + \epsilon_k + c_1 \epsilon_{k-1} + \ldots + c_{n_c} \epsilon_{k-n_c}.
\end{align*}
que também pode ser expresso por

\[
(1 - a_1 q^{-1} - \ldots - a_{n_a} q^{-n_a}) x_{k}
= (1 - c_1 q^{-1} - \ldots - c_{n_c} q^{-n_c}) \epsilon_{k} ,
\]

ou ainda $A(q) x_{k} = C(q) \epsilon_{k}$, com
\begin{align*}
A(q) & = (1 - a_1 q^{-1} - \ldots - a_{n_a} q^{-n_a}) \\
C(q) & = (1 - c_1 q^{-1} - \ldots - c_{n_c} q^{-n_c}).
\end{align*}


}


\frame{
\frametitle{Modelos ARIMA}

Um caso particular interessante dos modelos ARMA é obtido quando assume-se $n_i$ pólos integradores (I) no modelo da série temporal $x_k$, ou seja
\[
A(q) (1-q^{-1})^{n_i} x_{k} = C(q) \epsilon_{k}
\]

Note que
\begin{align*}
w_k & = (1-q^{-1}) x_{k} \\
	& = x_{k} - x_{k-1}
\end{align*}
pode ser vista como a derivada numérica de $x_{k}$.

Modelos ARIMA são tipicamente caracterizados pela tripla $\left(n_a, n_i, n_c \right)$ que define estrutura particular para representar uma série temporal.

}

\section{Ferramentas de software}% A biblioteca statsmodels


\begin{frame}[fragile]
%{A }

Biblioteca \textsf{statsmodels}: \url{https://www.statsmodels.org/stable/index.html}


\begin{itemize}
\item Geração de séries temporais;\\
\begin{lstlisting}[language=Python]
from statsmodels.tsa.arima_process import arma_generate_sample
\end{lstlisting}
%
\item Estimação dos parâmetros de modelos ARMA;\\
\begin{lstlisting}[language=Python]
from statsmodels.tsa.arima.model import ARIMA
\end{lstlisting}

\item Gráficos de análise.\\
\begin{lstlisting}[language=Python]
from statsmodels.tsa.stattools import pacf
from statsmodels.tsa.stattools import acf
from statsmodels.graphics.tsaplots import plot_pacf
from statsmodels.graphics.tsaplots import plot_acf
\end{lstlisting}
\end{itemize}

\end{frame}




\section{Exemplos didáticos}% A biblioteca statsmodels
\frame{
\frametitle{}

Vamos exercitar as ideias discutidas anteriormente através de um notebook \alert{ARMA\_models.ipynb}

}



\section{Considerações finais}

\frame{
\frametitle{}
\begin{enumerate}
\item 
\item 
\item 
\item 
\end{enumerate}
}

\end{document}






\begin{block}{Exemplo: Processo AR de segunda ordem com média nula}

Equação de diferenças: $x_k = 1.5 x_{k-1} - 0.7 x_{k-2} + \epsilon_k$

Diagrama de blocos:
%\vspace*{10pt}
\begin{center}
\begin{tikzpicture}
\sbEntree{E1}
\sbBloc[3]{Bloc}{$\dfrac{1}{1 -1.5q^{-1}+ 0.7q^{-1}}$}{E1}
\sbRelier[$\epsilon_k$]{E1}{Bloc}
\sbSortie[3]{S1}{Bloc}
\sbRelier[$x_k$]{Bloc}{S1}
\end{tikzpicture}
\end{center}
Pólos complexos: $q_{1,2}= 0.75 \pm 0.371i$
\end{block}

